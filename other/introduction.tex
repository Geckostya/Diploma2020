% У введения нет номера главы
\section*{Введение}
\renewcommand{\thefigure}{\arabic{figure}}
\subsection*{Актуальность и релевантные работы}

RISC-V --- это open-source архитектура процессора, которая с 2010 года разрабатывается для широкого круга применений. Данная разработка поддерживается многими крупными компаниями \cite{riscv:members}, в том числе такими как Alibaba, Google, Huawei, NVIDIA, Western Digital.

На текущий момент, платформа поддерживает исполнение множества программ и языков \cite{riscv:soft}, однако исполнение языка Java на RISC-V возможно только на двух исследовательских Java машинах, не подходящих для продуктовых решений из-за своей специфики и производительности.

OpenJDK --- это open-source проект, в котором разрабатывают эталонную \cite{openjdk_reference, openjdk:FAQ} продуктовую реализацию Java Development Kit (JDK), которая содержит Java Virtual Machine (JVM), библиотеку классов и прочие инструменты разработки, такие как компилятор javac. Имплементация JVM платформозависима, и её порты реализованы для многих современных архитектур \cite{openjdk:platforms}, но всё ещё нет порта для RISC-V.

\myfig{jvm.png}{Схематичное устройство JVM}{jvm}{0.6}

Упрощенное схематичное устройство JVM представлено на \figref{jvm}, сначала Java код компилируется в Bytecode инструкции (байткоды), которые попадают в JVM и там происходит их ассемблерная интерпретация в шаблонном интерпретаторе. Затем, когда статистика по исполнению будет собрана, активируется Just In Time (JIT) компилятор, оптимизирующий исполняемый код, однако периодически происходит деоптимизация кода, и исполнение снова возвращается в интерпретатор. Как видно, шаблонный интерпретатор --- ключевая и неотъемлемая часть JVM, наличие которой позволяет исполнять любой Java код.

При имплементации шаблонного интерпретатора необходимо продумать и реализовать его двоичный интерфейс приложения\footnote{Авторский перевод термина Application Binary Interface} (ABI), ключевую роль в котором занимают фреймы --- структуры данных, создаваемые на стеке при вызове функций. В Java фреймы содержат аргументы, локалы и прочие данные для интерпретации и выполнения ABI. Есть три ключевых перехода между функциями - из C в Java через входной фрейм, из Java в Java через Java фрейм и из Java в C через нативный Java фрейм.



\subsection*{Цель и задачи}

Данная работа ставит своей целью разработку и реализацию ABI шаблонного интерпретатора для порта OpenJDK на архитектуру процессора RISC-V. Для этого ставятся следующие задачи:

\begin{enumerate}
    \item Провести подготовительные работы для портирования OpenJDK, включающие в себя настройку рабочего окружения и реализацию базовых компонент порта OpenJDK.
    \item Разработать структуры фреймов и реализовать их для трех переходов между функциями в интерпретаторе.
    \item Определить набор данных, которые являются самыми востребованными в интерпретаторе и нуждаются в кэшировании в специально выбранных регистрах для уменьшения количества медленных обращений к памяти.
    \item Тестирование полученных результатов для проверки корректности реализации.
\end{enumerate}




\subsection*{Достигнутые результаты}

В рамках данной работы было сконфигурировано окружение, в том числе написаны различные скрипты для упрощения работы с ним, а также реализован генератор RISC-V ассемблера, необходимый для написания порта.

Были спроектированы структуры фреймов, основанные на нативном ABI C фреймов, что позволило сохранить единообразие системы, необходимое для корректной работы инструментов отладки и Runtime вызовов. Также входной, Java и нативный Java фреймы были реализованы в соответствии со всеми требованиями со стороны системы и нативных соглашений о вызовах. Их структура не подверглась значительным изменениям, в сравнении с другими портами портами, однако каждое решение по расположению данных на фрейме было обосновано.

Для выбора информации, нуждающейся в кэшированни, было проведено исследование с ручным и автоматическим сборами статистики, как для выявления полезности текущего кэширования данных, выбранного в других портах, так и для потенциального нахождения новых. Данные исследования показали острую необходимость кэширования ряда данных, в то время, как кэширование некоторых данных оказалось бесполезным или крайне сомнительным, требующем дальнейшего более индивидуального изучения. По ряду причин, включающих в себя большую трудозатратность, данное исследование не было полностью закончено и были намечены дальнейшие векторы развития в данном направлении, которые смогут помочь определить оптимальный набор данных для кэширования.

Для тестирования были выбраны несколько подходов, позволяющих, в первую очередь, проверить корректность работы с аргументами на фрейме, в том числе нативными, а также правильность создания и удаления фреймов со стека. Остальные аспекты данной работы не подлежат тщательному тестированию, поэтому их корректность была подтверждена лишь косвенно.

\subsection*{Структура работы}

В обзоре литературы представлены основные для данной работы источники информации в области как архитектуры RISC-V, так и проекта OpenJDK, а также рассмотрены альтернативы по замене литературы, необходимой для портирования OpenJDK.

В первой главе описаны работы по подготовке и настройке рабочего окружения, а также начало процесса портирования, а именно написание генератора ассемблера и неудачные попытки реализации вывода ошибок JVM.

Во второй главе представлены все факторы, учтенные в разработке фреймов, а также описание структур и реализации каждого фрейма.

В третьей главе представлены методы исследования по выбору информации для кэширования, а также полученные в ходе данного исследования результаты. 

В четвертой главе указаны используемые методы тестирования и представлены результаты по общей работе порта на данном этапе его разработки.

\renewcommand{\thefigure}{\arabic{section}.\arabic{figure}}