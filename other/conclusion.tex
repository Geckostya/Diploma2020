% У заключения нет номера главы
\section*{Заключение}

Главным результатом данной работы является реализация разработанного ABI шаблонного интерпретатора для порта OpenJDK на архитектуру процессора RISC-V. В рамках выполнения поставленной цели были достигнуты следующие задачи:

\begin{enumerate}
    \item Были настроены GNU Toolchain и QEMU эмулятор RISC-V, позволяющие компилировать и запускать проект с помощью специально написанных для этого скриптов, упрощающих всё взаимодействие. Успешность настроенного окружения была проверена посредством запуска zero порта OpenJDK. Был написан генератор RISC-V ассемблера, однако из-за неполадок в QEMU не удалось реализовать печать ошибок JVM.
    
    \item При проектировании фреймов были учтены все особенности платформы RISC-V и её нативный ABI, и хоть глобальная структура не изменилась кардинально в сравнении с другими портами, однако каждое решение было принято обосновано. Реализация генерации фреймов, их заполнении и поддержки была сделана с учетом необходимости высокой скорости работы приложения.
    
    \item Были разработаны ручная и автоматическая методики данных, которые были применены к трем источникам - инициализации JVM, Renaissance Suite и SPECjvm2008. Наиболее полезными оказались кэширующие регистры bcp, TOS, esp, threads, locals, fp и dispatch\_table. Регистр sender\_sp оказался малополезным, а TOC вовсе неиспользуемым, и от них необходимо избавиться, по мере технической возможности. Для проверки полезности monitors было получено недостаточно данных, поэтому необходимо углубить данное исследование в будущем. Кандидатов на замену данным регистрам также не удалось обнаружить, поэтому необходимо произвести более тщательное исследование.
    
    \item Несмотря на трудность тестирования нерабочего порта OpenJDK, было успешно сгенерировано и пройдено 1289 тестов на различные сигнатуры методов и передачу параметров. Генератор случайных арифметических выражений позволил протестировать работу с локалами. Рабочая часть инициализации Java также доказывает корректность произведенной работы.
\end{enumerate}

В качестве дальнейшей работы над проектом могут быть расширение и углубление исследования кэширования информации, которое позволит ответить на неразрешенные вопросы о полезности кэширования мониторов и нахождении замены иного малополезного кэширования. Также остались нереализованными процессы, вызываемые при создании фреймов, необходимые для профилирования и продвинутого сбора статистики исполнения. Общая структура ijava\_state, широко используемая в Runtime вызовах, одновременно содержит информацию, необходимую только в нативных фреймах, а также необходимую только в ненативных, что может быть соптимизировано, если тщательно проанализировать работу интерпретатора с данной структурой и разработать механизм, позволяющий избавиться от лишних данных в обоих случаях, который не помешает текущим процессам исполнения.