% В этом шаблоне используется класс spbau-diploma. Его можно найти и, если требуется, 
% поправить в файле spbau-diploma.cls
\documentclass[14pt]{spbau-diploma}
\begin{document}

% my@body14pt
\let\NeedsTeXFormat\ignore
\let\newcommand\renewcommand
\makeatletter
\input{size14.clo}%
\makeatother  
\let\newcommand\orignewcommand
\let\NeedsTeXFormat\origNeedsTeXFormat
\newgeometry{a4paper,top=20mm,bottom=20mm,left=20mm,right=15mm,nohead,includeheadfoot}

\thispagestyle{empty}
\begin{figure}[hbtp!]
    \centering
    \includegraphics[scale=1]{figs/title.png}
\end{figure}

\newpage

\newgeometry{a4paper,top=20mm,bottom=20mm,left=30mm,right=15mm,nohead,includeheadfoot}
\setcounter{page}{1}

\tableofcontents

% У введения нет номера главы
\section*{Введение}
\renewcommand{\thefigure}{\arabic{figure}}
\subsection*{Актуальность и релевантные работы}

RISC-V --- это open-source архитектура процессора, которая с 2010 года разрабатывается для широкого круга применений. Данная разработка поддерживается многими крупными компаниями \cite{riscv:members}, в том числе такими как Alibaba, Google, Huawei, NVIDIA, Western Digital.

На текущий момент, платформа поддерживает исполнение множества программ и языков \cite{riscv:soft}, однако исполнение языка Java на RISC-V возможно только на двух исследовательских Java машинах \cite{jikes, maxine}, не подходящих для продуктовых решений из-за низкой производительности, которая очень важна пользователям и разработчикам.

OpenJDK --- это open-source проект, в котором разрабатывают эталонную \cite{openjdk_reference, openjdk:FAQ} продуктовую реализацию Java Development Kit (JDK), которая содержит Java Virtual Machine (JVM), библиотеку классов и прочие инструменты разработки, такие как компилятор javac. Реализация JVM платформозависима, и её порты реализованы для многих современных архитектур \cite{openjdk:platforms}, но всё ещё нет порта для RISC-V.

\myfig{jvm.png}{Схематичное устройство JVM}{jvm}{0.6}

Упрощенное схематичное устройство JVM представлено на \figref{jvm}, сначала Java код компилируется в Bytecode инструкции (байткоды), которые попадают в JVM и там происходит их ассемблерная интерпретация в шаблонном интерпретаторе. Принцип данного интерпретатора заключается в сопоставлении каждому байткоду определенного набор ассемблерных инструкций, моделирующих его семантику. Затем, когда статистика по исполнению будет собрана, активируется Just In Time (JIT) компилятор, оптимизирующий исполняемый код, однако периодически происходит деоптимизация кода, и исполнение снова возвращается в интерпретатор. Как видно, шаблонный интерпретатор --- ключевая и неотъемлемая часть JVM, наличие которой позволяет исполнять любой Java код.

При имплементации шаблонного интерпретатора необходимо продумать и реализовать его двоичный интерфейс приложения\footnote{Авторский перевод термина Application Binary Interface} (ABI), ключевую роль в котором занимают фреймы --- структуры данных, создаваемые на стеке при вызове функций. В Java фреймы содержат аргументы, локалы и прочие данные для интерпретации и выполнения ABI. Есть три ключевых перехода между функциями - из C функции в Java метод через входной фрейм, из Java в Java через Java фрейм и из Java в C через нативный Java фрейм.



\subsection*{Цель и задачи}

Данная работа ставит своей целью разработку и реализацию ABI шаблонного интерпретатора для порта OpenJDK на архитектуру процессора RISC-V. Для этого ставятся следующие задачи:

\begin{enumerate}
    \item Провести подготовительные работы для портирования OpenJDK, включающие в себя настройку рабочего окружения и реализацию базовых компонент порта OpenJDK.
    \item Разработать структуры фреймов и реализовать их для трех типов переходов между функциями в интерпретаторе.
    \item Определить набор данных, которые являются самыми востребованными в интерпретаторе и нуждаются в кэшировании в специально выбранных регистрах для уменьшения количества медленных обращений к памяти.
    \item Тестирование полученных результатов для проверки корректности реализации.
\end{enumerate}




\subsection*{Достигнутые результаты}

В рамках данной работы было сконфигурировано окружение, в том числе написаны различные скрипты для упрощения работы с ним, а также реализован генератор RISC-V ассемблера, необходимый для написания порта. Данные результаты, в отличие от по результаты были достигнуты в рамках командной работы.

Были спроектированы структуры фреймов, основанные на нативном ABI C фреймов, что позволило сохранить единообразие системы, необходимое для корректной работы инструментов отладки и Runtime вызовов. Также входной, Java и нативный Java фреймы были реализованы в соответствии со всеми требованиями со стороны системы и нативных соглашений о вызовах. Их структура не подверглась значительным изменениям, в сравнении с другими портами, однако каждое решение по расположению данных на фрейме было обосновано.

Для выбора информации, нуждающейся в кэшированни, было проведено исследование с ручным и автоматическим сборами статистики, как для выявления полезности текущего кэширования данных, выбранного в других портах, так и для потенциального нахождения новых. Данное исследование показало острую необходимость кэширования ряда данных, в то время, как кэширование некоторых других оказалось менее полезным или крайне сомнительным, требующем дальнейшего более индивидуального изучения.

Для тестирования были выбраны несколько подходов, позволяющих, в первую очередь, проверить корректность работы с аргументами на фрейме, в том числе нативными, а также правильность создания и удаления фреймов со стека. Остальные аспекты данной работы не подлежат тщательному тестированию, поэтому их корректность была подтверждена лишь косвенно.

\subsection*{Структура работы}

В обзоре литературы представлены основные для данной работы источники информации в области как архитектуры RISC-V, так и проекта OpenJDK, а также рассмотрены альтернативы по замене литературы, необходимой для портирования OpenJDK.

В первой главе описаны работы по подготовке и настройке рабочего окружения, а также начало процесса портирования, а именно написание генератора ассемблера и печати ошибок JVM.% и неудачные попытки реализации вывода ошибок JVM.

Во второй главе представлены все факторы, учтенные в разработке фреймов, а также описание структур и реализации каждого фрейма.

В третьей главе представлены методы исследования по выбору информации для кэширования, а также полученные в ходе данного исследования результаты. 

В четвертой главе указаны используемые методы тестирования и представлены результаты по общей работе порта на данном этапе его разработки.

В конце работы представлен глоссарий с определениями самых важных терминов, специфичных для данной области.

\renewcommand{\thefigure}{\arabic{section}.\arabic{figure}}

\section*{Аннотация}

Активно развивающаяся архитектура процессора RISC-V, имеющая большую поддержку от лидирующих компаний со всего мира, всё ещё не имеет реализации OpenJDK --- эталонной и эффективной среды для разработки и исполнения Java, одного из самых популярных языков программирования. Важнейшим элементом исполнения Java является шаблонный интерпретатор, для реализации которого необходимо разработать ABI, включающий в себя структуры фреймов на стеке и использование регистров для кэширования информации. В ходе данной работы были разработаны и реализованы структуры фреймов, основанные на нативном ABI фреймов RISC-V, также было проведено исследование, основанное на полуавтоматически собранной статистике исполнения Java кода, в результате которого были выявлены наиболее и наименее нуждающиеся в кэшировании данные. Также были найдены и использованы способы протестировать результаты на ещё неполностью рабочем порте.

\vspace*{1em}

Ключевые слова: RISC-V, OpenJDK, шаблонный интерпретатор, ABI, фрейм, кэширование.

\section*{Abstract}

The actively developing RISC-V instruction set architecture, which has great support from leading companies from around the world, still does not have an OpenJDK implementation --- a reference and efficient environment for developing and executing Java, which is one of the most popular  prog-\newline ramming languages. The most important element of Java execution is the template interpreter, for the implementation of which it is necessary to develop an ABI that includes stack frame structures and the use of registers for information caching. In the course of this work, frame structures based on the native ABI of RISC-V frames have been developed and implemented, as well as a study based on semi-automatically collected statistics on the execution of Java code, as a result of which the most and the least caching data have been identified. The solution has been found and used in testing the results on a still incomplete working port.

\vspace*{1em}

Keywords: RISC-V, OpenJDK, template interpreter, ABI, frame, caching.

\section*{Обзор литературы}

\subsection*{Спецификация Java машины}

Главным и официальным источником информации о Java машине и её устройстве является The Java Virtual Machine Specification (JVMS) \cite{jvms}. В~нашей работе рассматривается тринадцатая версия JVMS, как самая актуальная на момент начала работы, однако, интересующие нас разделы являются основополагающими, и не менялись долгие годы. 
В~разделе 2.3 JVMS описываются типы данных в Java, их размеры и строгая спецификация диапозона допустимых значений. Данный раздел крайне полезен при имплементации трансляции Java значений в значения языка C, и обратно, чтобы не нарушить спецификацию и внутреннее устройство этих языков. Раздел 2.6 является одним из самых основных для данный работы, так как описывает устройство Java фреймов, разработка которых является ключевым этапом в разработке ABI для \gls{порт}[а] Java машины. В этой главе описывается, что фреймы создаются при каждом вызове метода, а текущим фреймом считается тот, что был создан при вызове текущего метода на исполнение. На фреймах хранятся аргументы, локалы, и стек вычислений, а также прочая информация, требуемая для реализации. Аргументы и локалы образуют единый список переменных функции и индексируются совместно таким образом, что если у нас имеется n аргументов и m локалов, то аргументы будут иметь индексы от 0 до n - 1, а локалы от n до n + m - 1, далее, при исполнении байткодов, логическая грань между переменными и локалами стирается, и всё обращение к ним происходит единообразно по индексу. Информация о количестве локалов, аргументов и размере стека вычислений рассчитывается во время компиляции, и содержится в структуре исполняемого метода.

Стоит отметить, что данная спецификация описывает требования, предъявляемые к системе, но не описывает детали имплементации, и тем более она не описывает устройство OpenJDK, понимание работы которого необходимо для его портирования.


\subsection*{OpenJDK}

Полной и качественной документации по внутреннему устройству OpenJDK не существует, однако весь код проекта находится в открытом доступе, частично задокументирован в виде комментариев к классам и методам, а остальную семантику выполнения процессов интерпретации можно вычитать напрямую из кода \cite{hotspot}.

Весь проект разделен на общую и платформозависимые части, которые относятся к конкретной операционной системе и процессору. Кодовая база проекта насчитывает сотни файлов в общей части, и десятки для каждой платформы, а каждый файл может достигать нескольких тысяч строк в длину, что делает изучение всего проекта в короткие сроки маловозможным.

Большинство портов OpenJDK написаны на основе других, и по коду, или оставшимся от иной платформы комментариям, можно проследить данное заимствование, что, порой, может усложнять восприятие. Также встречаются редкие расхождения комментариев и имплементации, что так же сбивает с толку при прочтении. При анализе различных портов можно также заметить некоторые общие практики, относящиеся, по большей части, к организации кода и внутренних структур, и, наоборот, сугубо платформоспецифичные, которые уникальны, и не поддаются общей характеристике.


\subsubsection*{Спецификация RISC-V}

Основным документом по работе с архитектурой RISC-V является его официальная спецификация \cite{riscv:spec}, выложенная у них на сайте. В данном документе описаны кодировки и описания всех инструкций и регистров, а также специфику вычислений, внутренней работы и прочих аспектов, необходимых для разработчиков, пишущих низкоуровневые приложения под данную платформу. 

Остальную документацию по связанным с RISC-V окружением и компонентами можно найти на их официальной странице в GitHub \cite{riscv:github}. Например, данная страница содержит репозиторий \cite{riscv:convention}, в котором описана конвенция о вызовах~C, описывающая семантику выкладки аргументов в регистры и на стек при вызове C~функций. Также, на данной странице можно найти репозиторий \cite{riscv:asm} с более подробной документацией некоторых ассемблерных особенностей данной платформы.


\subsection*{Выводы}

По данной тематике не удалось найти литературы, описывающей решение конкретной проблемы продумывания и имплементации ABI для OpenJDK, либо портирования OpenJDK на другие платформы, однако в данной среде есть немалое количество документов и спецификаций, посвященных системам, с которыми необходимо плотно работать. Необходимую литературу по OpenJDK можно заменить кодом и комментариями к нему, которые, однако, порой бывают невыверены или плохого качества. Остальная найденная литература по OpenJDK не имеет отношения к данному проекту прямым или косвенным образами, и потому не нуждается в упоминании.



\section{Подготовка окружения для написания OpenJDK порта}

\subsection{Настройка программ для разработки RISC-V}

\subsection{Настройка и реализация ключевой функциональности порта}


\section{Разработка и реализация фреймов}

Структуры фреймов в OpenJDK являются ключевым элементом ABI всего интерпретатора и определяют такие вещи, как выбор и способ сохранения необходимой информации во время интерпретации методов; способ взаимодействия Java методов, как друг с другом, так и с JNI и Runtime методами, вызываемыми в процессе интерпретации; семантику взаимодействия со стеком вычислений, локальными значениями и аргументами. Правильное создание данных структур должно опираться как на нативную реализацию структуры фреймов, чтобы сохранить совместимость с нативными вызовами и инструментами отладки, так и на создание и использование этих структур в интерпретации, чтобы обеспечить максимально быструю работу приложения.


\subsection{Разработка структуры фреймов}

Первой задачей разработки структуры фреймов было нахождение или составление описания нативных фреймов для языка C. Данные, найденные в онлайн лекции Parallel \& Distributed Operating Systems Group \cite{lecture:frames}, не являются официальным документом от разработчиков компилятора языка C, поэтому их необходимо было проверить, для чего использовался метод обратной инженерии. Были написаны различные программы на языке C, в которых были отражены ключевые сценарии вызовов функций, передачи аргументов и работы с локальными переменными. Далее, данный код был скомпилирован с помощью RISC-V GNU GCC \cite{riscv:gnu} компилятора в ассемблерный код RISC-V, после чего в нём были выявлены места, отвечающие за создание и заполнение фреймов. Семантика данного кода полностью соответствовала найденной структуре, и было принято решение использовать её в качестве основы для создания структуры Java фреймов. 

Проектирование структуры Java фреймов для RISC-V было основано на структуре фреймов для архитектуры PowerPC, однако существенные различия в нативных фреймах данных архитектур оказали соответственное влияние, а именно: нативный ABI RISC-V фреймов содержит только адрес в памяти, указывающий на начало \textbf{предыдущего} фрейма(началом фрейма будем считать последний байт перед этим фреймом), и адрес возврата для \textbf{текущей} функции, в то время как у PowerPC хранится начало \textbf{текущего} фрейма, адрес возврата \textbf{вызываемой} функции, имеется зарезервированное место под первые 8 аргументов и все эти значения лежат внизу фрейма, а не наверху, как в RISC-V.
Данные отличия, а также наличие регистра fp\footnote{Образовано от frame pointer}, указывающего начало текущего фрейма, повлекли изменения в реализации методов, которые помогают осуществить просмотр данных фрейма и навигацию между фреймами во время исполнения внутренних Runtime вызовов.

    Одной из особенностей архитектуры RISC-V является обязательное выравнивание вниз на 16 байт значения в регистре sp\footnote{Образовано от stack pointer}, которое должно указывать на последний байт текущего фрейма. Во время исполнения кода необходимо соблюдать данное выравнивание, что означает выбор одного из трех решений:
\begin{enumerate}
    \item Применять выравнивание ко всем данным, выкладываемым на стек.
    \item Помнить невыровненное смещение относительно sp, при работе со стеком сдвигать и его, и значение в sp.
    \item Создавать фрейм фиксированного размера и выкладывать все данные внутри него.
\end{enumerate}

Первый вариант нам не подходит из-за того, что размер данных на стеке увеличится примерно в 2 раза, что является неоправданной тратой памяти. Решено было выбрать третий, так как хоть он и требует небольших накладных расходов памяти на поддержание места под значения, которые не всегда необходимо сохранять в памяти, однако, в отличие от второго, он не усложняет работу со значениями на стеке, что может существенно замедлить процесс интерпретации.
% TODO написать про то, что остальные поля заполнялись и менялись исходя из нуждн реализации
% TODO вставить ссылку на приложения со структурой фрейма

\subsection{Реализация входного фрейма}

Самой первой точкой входа в интерпретатор является генерация входного фрейма, который, как и все остальные фреймы, написан с помощью нашего ассемблерного генератора. 

Для создания входного фрейма необходимо рассчитать и выделить под него место на стеке и заполнить все структуры, отображенные на \figref{входной_фрейм}, актуальными данными. Для заполнения нативного ABI необходимо сохранить в верхние два слова значения регистров ra\footnote{Образовано от return adress} и fp, которые указывают на адрес возврата функции и на начало предыдущего фрейма соответственно. "Сохраненные регистры" содержат в себе значения всех неизменяемых регистров, кроме fp и sp, это необходимо для соблюдения конвенции о вызовах C \cite{riscv:convention}, а именно, соблюдение правила о том, что неизменяемые регистры должны иметь одно и то же значение до и после вызова функции. Значение sp регистра будет сохранено в регистре fp, а значение fp мы сохранили в нативном ABI, поэтому их значения не нужно сохранять дополнительно. Локальные данные содержат несколько значений, доступ к которым может потребоваться во время интерпретации, а исходящие аргументы - это аргументы, пришедшие из C вызова, которые необходимо перекопировать на стек для совместимости разных вызовов.
 
\myfig{entry_frame.png}{Структура входного фрейма}{входной_фрейм}{0.5}
 
Во время заполнения данных структур возникали трудности с расставлением правильных смещений относительно fp регистра для точного определения начала структур. Также было необходимо правильно выбрать регистры для всех промежуточных расчетов, чтобы не получилось перекрытия данных. Помимо этого, на данном этапе появилась необходимость загрузки констант в регистры, так как архитектура RISC-V не предусматривает загрузку 64-битной константы в регистр одной инструкцией, и для этих целей нужно написать свой генератор, который по константе сгенерирует оптимальный набор инструкций, загружающих в регистр эту константу. Данный генератор уже написан в GCC, однако найти этот код в огромной системе не получилось, поэтому было решено написать собственный.

\TODO{Нужно ли писать более точные технические подробности про генератор?}

\TODO{Нужно ли указать, что генератор писался коллективно?}



\subsection{Реализация Java фрейма}
При вызове Java метода происходит создание Java фрейма. На предыдущем фрейме должны быть выложены аргументы, а регистр s7, который мы выбрали и назвали esp\footnote{Образовано от expression stack pointer}, и который указывает на первый свободный слот стека вычислений\footnote{Авторский перевод термина operand stack из JVMS}, указывает на место, куда необходимо выложить локалы. В зависимости от размера стека вычислений предыдущего фрейма, занятого на нём места и количества аргументов и локалов текущего метода, предыдущий фрейм необходимо увеличить, чтобы вместились все данные, либо уменьшить, чтобы не занимать лишнего места на стеке, которое мы не будем использовать во время интерпретации текущего метода, для этого нужно соответствующим образом сдвинуть регистр, указывающий на начало текущего фрейма, запомнив конец предыдущего фрейма до смещения в специально отведенном месте не стеке, чтобы восстановить его первозданный вид после выхода из метода.

Трудности в данной работе возникали, как и для входного фрейма, в правильном расчете всех размеров и сдвигов. В отличие порта PowerPC, у нас нет необходимости перекопировать структуры нативного ABI при изменениях родительского фрейма, а также у нас нет необходимости хранить на последнем фрейме место под 8 аргументов для Runtime вызовов. Данные отличия помогли изменить структуру расчетов размеров фреймов, сделав её проще и понятнее для восприятия, не жертвуя количеством инструкций, необходимых для расчетов.

\myfig{Ijava_frame.png}{Структура Java фрейма}{java_фрейм}{0.45}

Были приняты попытки по значительному изменению структуры Java фрейма, изображенного на \figref{java_фрейм}, однако они не увенчались успехом. Для упрощения обращения к аргументам и локалам функции, которые находятся на предыдущем фрейме, можно зарезервировать регистр, который будет указывать на начало этих данных. Однако можно заметить, что эти данные расположение прямо перед началом текущего фрейма, таким образом, что регистр fp указывает на конец этих данных с учтенным смещением на 16 байт. Если полностью инвертировать порядок аргументов и локалов, то можно добиться того, чтобы первый fp указывал ровно на первый аргумент, и от него можно было бы получить доступ ко всем остальным, однако, в связи со спецификацией Java, аргументы выкладываются на стек с помощью обычных байткодов, выкладывающих любые значения на стек, что делает невозможным контроль данного процесса. Единственным возможным решением в данном случае является копирование аргументов на новое место и в обратном порядке при вызове функции, однако это слишком дорогая операция по количеству инструкций на обращение к стеку, и, в итоге, более правильным, в целях оптимизации, решением будет выделение дополнительного регистра для упрощения адресации.

Были приняты попытки изменить расположение мониторов\footnote{Специальные структуры, отвечающие за механизмы синхронизации многопоточных вычислений}, в связи с тем, что исходя из статических данных, полученных из Java компилятора, нельзя вычислить точное количество места, требуемого под расположение мониторов, что приводит к копированию всех данных под мониторами, при выкладывании их на стек и сопутствующим расширении фрейма. Исходя из данных рассуждений следует вывод, что необходимо располагать мониторы как можно ближе к концу фрейма, чтобы как можно меньше данных пришлось копировать. При рассмотрении рисунка \ref{fig:java_фрейм}, следовательно, можно заметить, что передвигать мониторы на первую или вторую позицию будет менее рационально с позиции уменьшения количества копируемых данных. Исходящих аргументов и локалов на фрейме при выкладывании монитора ещё нет, в связи с тем, что мониторы выкладываются только на текущий фрейм, а значит остаются только позиции до или после стека вычислений, однако, исходя из того, что размер стека вычислений не учитывает количество локалов вызываемых функций, по причине невозможности данных подсчетов, при выкладывании мониторов после стека вычислений мы не только лишимся возможности быстрого изменения размера стека при вызове, чтобы занять свободное место стека вычислений, но и разорвем непрерывность аргументов и локалов на стеке, что увеличит количество операций, требуемых на работу с ними. Таким образом, в особенности учитывая тот факт, что мониторы выкладываются преимущественно при пустом стеке вычислений, их текущая позиция является оптимальной.



\subsection{Реализация нативного Java фрейма}
JNI методы написаны на языках C или C\texttt{++}, и для их вызова необходимо соблюдать конвенцию о вызовах C. Для решения этой задачи создается специальная структура нативного Java фрейма, отличающаяся от своей ненативной реализации отсутствием стека вычислений и нативным форматом исходящих аргументов. Данный фрейм является лишь прослойкой между Java и С языками, служащий лишь в качестве корректного форматирования аргументов, выкладывании монитора для синхронизированных вызовов и сохранением необходимых данных в структуре ijava\_state.


\myfig{native_frame.png}{Структура нативного Java фрейма}{нативный_java_фрейм}{0.6}


В OpenJDK копирование Java аргументов в C формат представлено двумя способами - быстрым и медленным. Данный выбор переключается с помощью выставления специального флага при запуске JVM, которая по умолчанию использует быструю реализацию. Медленное копирование аргументов читает сигнатуру метода непосредственно в ассемблерном коде, и по ней происходит копирование каждого типа данных из сигнатуры надлежащим образом. Данный метод позволяет сгенерировать код один раз для всех вызовов метода, однако чтение и разбор сигнатуры из асемблера накладывает дополнительные вычислительные затраты на данный код, а также требуют двух Runtime вызовов для взятия необходимых значений из метода. Быстрое копирование аргументов подразумевает генерацию уникального кода для каждой сигнатуры, который будет максимально быстро копировать аргументы, зная про формат сигнатуры на момент генерации. В рамках данной работы было реализовано оба этих метода.

В рамках генерации нативного Java фрейма делается значительно больше проверок и обработок исключительных ситуаций, чем при генерации ненативного фрейма. Помимо этого, появляется необходимость обработки результата JNI метода, чтобы он соответствовал значениям Java типов, специфицированных в JVMS. Так, например, значения булевого\footnote{Тип данных, имеющий только два значения - ложь и истина} типа в Java должны содержать только $0$ или $1$, что является ложью или истиной соответственно, в то время как в спецификации C указано, что $0$ соответствует ложному значению, а любое другое число истинному. В связи с подобными ограничениями были придуманы и реализованы минимальные по сложности инструкции переводов значений C типов в значения, удовлетворяющие Java спецификации и внутренней работе порта.

\TODO{Ссылки на код в приложении? Объяснение примера с булевым значением?}




\section{Кэширование информации интерпретатора}

\subsection{Использование регистров для кэширования в портах}

\subsection{Выбор источников для сбора статистики}

\subsection{Сбор статистики}

\subsection{Обработка результатов}


\section{Тестирование результатов}

В OpenJDK есть огромное количество тестов, в достаточно полной мере тестирующие все аспекты системы, однако их применение невозможно на данном этапе разработки, в связи с тем, что порт ещё не дописан и не может до конца проинициализировать JVM, необходимую для корректной работы тестов.
Даже если сделать обходной путь для запуска этих тестов, то невозможно будет оценить результаты тестирования и каждый проваленный тест будет требовать отдельного и тщательного изучения, что потребует слишком большого количества времени.

В связи с данными ограничениями приходилось прибегать к дополнительным методам тестирования, о которых и будет рассказано в данной главе.

\subsection{Объект тестирования}

\subsection{Этапы тестирования}

\subsection{Выводы}



% У заключения нет номера главы
\section*{Заключение}

Главным результатом данной работы является реализация разработанного ABI шаблонного интерпретатора для порта OpenJDK на архитектуру процессора RISC-V. В рамках выполнения поставленной цели были достигнуты следующие задачи:

\begin{enumerate}
    \item Были настроены GNU Toolchain и QEMU эмулятор RISC-V, позволяющие компилировать и запускать проект с помощью специально написанных для этого скриптов, упрощающих всё взаимодействие. Успешность настроенного окружения была проверена посредством запуска zero порта OpenJDK. Был написан генератор RISC-V ассемблера, однако реализация печати ошибок JVM была отложена из-за неполадок в QEMU.
    
    \item При проектировании фреймов были учтены все особенности платформы RISC-V и её нативный ABI, и хоть глобальная структура не изменилась кардинально в сравнении с другими портами, однако каждое решение было принято обоснованно. Реализация генерации фреймов, их заполнения и поддержки были сделаны с учетом необходимости высокой скорости работы приложения.
    
    \item Были разработаны ручная и автоматическая методики сбора данных, которые были применены на порте x86\_64 к трем источникам - инициализации JVM, Renaissance Suite и SPECjvm2008. Наиболее полезными оказались кэширующие регистры bcp, TOS, esp, threads, locals, fp и dispatch\_table. Регистр sender\_sp оказался малополезным, а TOC вовсе неиспользуемым, и от них необходимо избавиться по мере технической возможности.
    
    \item Несмотря на трудность тестирования нерабочего порта OpenJDK, были успешно сгенерированы и пройдены 1289 тестов на различные сигнатуры методов и передачу параметров. Генератор случайных арифметических выражений позволил протестировать работу с локалами. Рабочая часть инициализации Java также доказывает корректность произведенной работы.
\end{enumerate}

В качестве дальнейшей работы над проектом могут быть выполнены работы по расширению и углублению исследования кэширования информации, которые позволят ответить на неразрешенные вопросы о полезности кэширования мониторов и нахождении замены иного малополезного кэширования. Помимо этого, текущий сбор статистики показывает корректные данные только при однопоточном исполнении, что может влиять на показатели для данных, отвечающих за процессы синхронизации, поэтому необходимо найти способ собирать данную статистику в параллельных приложениях. Также остались нереализованными процессы, вызываемые при создании фреймов, необходимые для профилирования и продвинутого сбора статистики исполнения. Общая структура ijava\_state, широко используемая в Runtime вызовах, одновременно содержит информацию, необходимую только в нативных фреймах, а также необходимую только в ненативных, что может быть соптимизировано, если тщательно проанализировать работу интерпретатора с данной структурой и разработать механизм, позволяющий избавиться от лишних данных в обоих случаях, который не помешает текущим процессам исполнения.


\glsaddall
% altlist, longragged, altlong4col 
\printnoidxglossary[title=Глоссарий, toctitle=Глоссарий, style=altlist,nonumberlist]

\bibliographystyle{ugost2008ls}
\bibliography{diploma.bib}
\end{document}
